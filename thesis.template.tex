% created by Sebastian Nemak - der-basti.com
\documentclass[
	12pt % default font size
	,a4paper % classic A4 page, default onesided (see geometry package options)
	,headings=normal % smarter headings
	,toc=graduated % table of content with indent (not '=flat')
]{scrreprt} % KOMA-Script report document typ

% -----------------------------------------------------------------------------
% packages, settings, \renewcommand and meta data
% -----------------------------------------------------------------------------

\usepackage[utf8]{inputenc} % set utf-8 input encoding
\usepackage[T1]{fontenc} % 8-bit encoding for languages with accented characters
\usepackage[ngerman,english]{babel} % to use (new) german and english language
\usepackage{lmodern} % use vector font
\usepackage[top=27mm, left=35mm, right=20mm, bottom=30mm, headsep=10mm, footskip=12mm]{geometry} % onesided with binding
\usepackage{hyperref} % han­dle cross-ref­er­enc­ing
\usepackage{amsmath,amsfonts,amssymb} % pro­vides math­e­mat­i­cal fea­tures
\usepackage{setspace} % set­ting the spac­ing be­tween lines
\usepackage[german=swiss]{csquotes} % pro­vides ad­vanced fa­cil­i­ties for in­line and dis­play quo­ta­tions e.g. \enquote{Text}

\usepackage{blindtext} % OPTINAL: for demo text purposes

\hypersetup{ % pdf options
	pdfauthor = {Your Name}
	,pdftitle = {Latex Template}
	,pdfsubject = {bachelor or master thesis subject}
	,pdfcreator={\LaTeX}
	,pdfkeywords = {Stichwort1, Stichwort2 ...}
	,pdfproducer = {pdfTeX \the\pdftexversion.\pdftexrevision}
	,breaklinks=true,unicode=false, pdftoolbar=true, pdfmenubar=true, pdffitwindow=false, pdfstartview={FitH},pdfnewwindow=true
	,colorlinks=true, linkcolor=black, citecolor=black, filecolor=magenta, urlcolor=black
}
\setcounter{tocdepth}{3} % toc depth level - list subsubsection
\setcounter{secnumdepth}{3} % toc depth level - number subsubsection
\setlength\parindent{0pt} % 0/no auto paragraph indent

\author{Your Name}
\title{Latex Template}
\subtitle{bachelor or master thesis subject}
\date{\today}

\begin{document}

\selectlanguage{ngerman}
% Vorspann
\renewcommand{\thesection}{\Roman{section}} \renewcommand{\theHsection}{\Roman{section}}
\pagenumbering{Roman}

% -----------------------------------------------------------------------------
% title page
% -----------------------------------------------------------------------------

\maketitle

% -----------------------------------------------------------------------------
% abstract
% -----------------------------------------------------------------------------

\section*{Zusammenfassung} % use section to print de and en on one page
\addcontentsline{toc}{chapter}{Zusammenfassung}
\blindtext

\section*{Abstract}
\selectlanguage{english}
\blindtext
\selectlanguage{ngerman}

\newpage

% -----------------------------------------------------------------------------
% acknowledgments
% -----------------------------------------------------------------------------

\section*{Danksagung}

Für die Unterstützung ...\par

% -----------------------------------------------------------------------------
% table of ...
% -----------------------------------------------------------------------------

\tableofcontents

\newpage
\onehalfspacing % for the rest document
\renewcommand{\thesection}{\arabic{section}}
\renewcommand{\theHsection}{\arabic{section}}
\pagenumbering{arabic}
\setcounter{section}{0}
\setcounter{page}{1}

% -----------------------------------------------------------------------------
% your content
% -----------------------------------------------------------------------------

\blinddocument

\chapter{Mathematische Formeln}\label{cha:mathematischeFormeln}

\blindmathpaper

\chapter{Formatierungen}\label{cha:formatierungen}

\blindtext

\section{Text}\label{sec:text}

Zum guten Style gehört:
\begin{itemize}
	\item Versuche alle Warnungen beim erstellen zu beheben
	\item Zu einem Chapter, Section, Grafik, Tabelle, ... gehört immer ein \textit{caption} und \textit{label}
\end{itemize}

Umlaute können einfach benutzt werden z.B. äÖü. Hingegen mu{\ss} das {\ss} mit \textbackslash ss umschrieben werden.\par

Text: \textbf{Bold}, \textit{italic}, \emph{Emphasis}.\par

\subsection{Verweise}\label{ssec:verweise}

Referenz: ref: \ref{tab:matrix}, autoref: \autoref{tab:matrix}\par

Fußnote \footnote{Text in der Fußnote}.\par

\enquote{Dies ist ein Zitat direkt in der Zeile.}\par

Eine klassische URL \url{https://google.com}. % url syntax is a little bit different between package hyperref and url
Natürlich kann eine URL auch so aussehen \href{https://google.com}{Google}.\par

Cite: %TODO

\subsection{Ausrichtung}\label{ssec:ausrichtung}

\begin{flushleft}
left
\end{flushleft}

\begin{center}
center
\end{center}

\begin{flushright}
right
\end{flushright}

\subsection{Grö{\ss}en}\label{ssec:groessen}

\begin{tiny}
• tiny
\end{tiny}

\begin{scriptsize}
• scriptsize
\end{scriptsize}

\begin{footnotesize}
• footnotesize
\end{footnotesize}

\begin{small}
• small
\end{small}

\begin{normalsize}
• normalsize
\end{normalsize}

\begin{large}
• large
\end{large}

\begin{Large}
• Large
\end{Large}

\begin{LARGE}
• LARGE
\end{LARGE}

\begin{huge}
• huge
\end{huge}

\begin{Huge}
• Huge
\end{Huge}

\subsection{Quellcode}\label{quellcode}

%TODO

\section{Grafiken}\label{sec:grafiken}

%TODO

\section{Tabellen}\label{sec:tabellen}

\begin{table}[ht]
	\centering
	\label{tab:matrix}
	\begin{tabular}{ | l | c | r | p{5cm} | }
		\hline
		\textbf{Left (l)} & \textbf{Center (c)} & \textbf{Right (r)} & \textbf{p 5cm breit}\\
		\hline
		1 & 2 & 3 & 4\\
		\hline
		5 & 6 & 7 & 8\\
		\hline
	\end{tabular}
	\captionbelow[Matrix Beschriftung im ToT-Verzeichnis]{Matrix Beschriftung}
\end{table}

% -----------------------------------------------------------------------------
% appendix
% -----------------------------------------------------------------------------

\appendix
\chapter{Appendix}\label{cha:appendix}
\pagenumbering{Roman}
\setcounter{page}{1}

\section*{This section appendix are not listed in the toc}\label{sec:appendixNotInToc}
\section{First appendix}\label{sec:firstAppendix}

\newpage

\section{Second appendix}\label{sec:secondAppendix}

\end{document}